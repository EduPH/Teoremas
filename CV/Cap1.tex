\chapter{Nociones básicas de funciones sobre espacios euclídeos}

\begin{itemize*}
\item El espacio euclídeo $\mathbb{R}^n=\{(x_1,\dots,x_n):x_i\in \mathbb{R}\}$ espacio topológico sii $\mathbb{R}$-espacio vectorial.

\item Dotada de distancia euclídea, compatible, producto escalar, etc.

\item Topología euclídea: abiertos, cerrados (por sucesiones convergentes), compactos.

\item Noción de continuidad. $f:\mathbb{R}^n \supseteq U\rightarrow \mathbb{R}^n$, $U$ abierto euclídeo. (Manda sucesiones convergentes en sucesiones convergentes)

\item La imagen continua de un compacto es compacto, $f(K)$ es compacto si $K$ es compacto. Análogo por un conexo o arco-conexo. 
\end{itemize*}


\section{Diferenciabilidad}

Sea $f:\mathbb{R}\rightarrow \mathbb{R}$ es diferenciable en $a\in \mathbb{R}$ si $\exists f'(a)\in \mathbb{R}$ tal que 
$$\lim_{h\rightarrow 0} \frac{f(a+h)-f(a)}{h}=f'(a)$$

Es decir, existe una transformación lineal $\lambda:\mathbb{R}\rightarrow \mathbb{R}$ tal que
$$\lim_{h\rightarrow 0} \frac{f(a+h)-f(h)-\lambda(h)}{h}=0$$

Y se puede generalizar a espacios euclídeos cualesquiera:

Sea $f:\mathbb{R}^n\rightarrow \mathbb{R}^m$ diferenciable en $a\in \mathbb{R}^n$ si $\exists \lambda : \mathbb{R}^n\rightarrow \mathbb{R}^m$ homomorfismo de espacios vectoriales, tal que

$$\lim_{h\rightarrow 0} \frac{|f(a+h)-f(a)-\lambda(h)|}{h}$$
$$\lambda=Df(a), \lambda \text{ es único}$$


\subsection{Regla de la cadena}

$\mathbb{R}^n\xrightarrow{f} \mathbb{R}^m \xrightarrow{g} \mathbb{R}^p$, entonces
$$D(g\circ f)(a)=Dg(f(a))\circ Df(a) $$

\begin{enumerate}
\item $f$ es constante, entonces $Df(a)=0$.
\item $f$ es lineal, $Df=f$.
\item $f= \pi_i \circ f_i \Rightarrow \exists Df(a) \Leftrightarrow \forall i Df_i(a)$. 
\end{enumerate}

\begin{Cor}
\begin{enumerate}
\item $D(f+g)(a)=Df(a)+Dg(a)$
\item $D(fg)=g(a)Df(a)+f(a)Dg(a)$
\end{enumerate}
\end{Cor}

\subsection{Derivadas parciales}


Si $f:\mathbb{R}^n \rightarrow \mathbb{R}^m$, $a\in \mathbb{R}^n$.

$$\exists D_if(a)=\lim_{h\rightarrow 0} \frac{f(a^1,\dots a^i+h,\dots , a^n)-f(a^1,\dots , a^n)}{h}$$

es la i-ésima derivada parcial para todo $a\in \mathbb{R}^n$.


Si $f$ es continua y $\exists $ todas las derivadas parciales de todos los órdenes, entonces diremos que $f$ es $C^\infty $.


\begin{Teo}
Sea $f:\mathbb{R}^n\rightarrow \mathbb{R}^m$ diferenciable en $a\in \mathbb{R}^n$, entonces existe $\forall i\le i \le m, 1\le j \le n$, $D_jf^i(a)$ y $f'(a)=(D_jf^i(a)):\mathbb{R}\rightarrow \mathbb{R}^m$ homomorfismo de espacios vectoriales. 
\end{Teo}

\begin{Teo}
Si $f:\mathbb{R}^n\rightarrow \mathbb{R}^m$ y $\exists (D_jf^i(a))$, entonces $f$ es diferenciable en $a$ si $D_jf^i$ son continuas en $a$, $\forall i,j$.  (Se dice que $f$ es continuamente diferenciable). 
\end{Teo}

\subsection{Regla de la cadena extendida}

Sea $g_1,\dots, g_m:\mathbb{R}^m \rightarrow \mathbb{R}$ continuamente diferenciables en $a\in \mathbb{R}^n$.  Sea $f:\mathbb{R}^m\rightarrow \mathbb{R}$ y sea $F:\mathbb{R}^n\rightarrow \mathbb{R}, x\mapsto F(x)=f(g_1(x),\dots , g_m(x))$. $\mathbb{R}^n\xrightarrow{G} \mathbb{R}^m \xrightarrow{f} \mathbb{R}$. Entonces, $D_iF(a)=\sum_{j=1}^m D_jf(g_1(a),\dots , g_m(a))\cdot D_ig_j(a)$.

Esto quiere decir que 
$$D(f\circ G)(a)=Df(G(a))\circ DG(a)=(D_1f(G(a)),\dots, D_mf(G(a))) (DG'(a)\dots DG^m(a))^t$$