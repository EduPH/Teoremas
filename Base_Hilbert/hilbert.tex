\documentclass{article}
\usepackage[spanish]{babel}
\usepackage[utf8]{inputenc}
\usepackage[T1]{fontenc}
\usepackage{graphicx}
\usepackage{fancyvrb}              % Verbatim extendido
\usepackage{makeidx}               % Índice
\usepackage{amsmath}               % AMS LaTeX
\usepackage{amsthm} 
\usepackage{latexsym}
\usepackage{vmargin}
\newtheorem{teor}{Teorema}
\newtheorem{nota}{Nota}
\setpapersize{A4}
\setmargins{2.5cm}       % margen izquierdo
{1.5cm}                        % margen superior
{16.5cm}                      % anchura del texto
{23.42cm}                    % altura del texto
{10pt}                           % altura de los encabezados
{1cm}                           % espacio entre el texto y los encabezados
{0pt}                             % altura del pie de página
{2cm}                           % espacio entre el texto y el pie de página
\begin{document}

\title{Teorema de la base de Hilbert}
\maketitle

\large

\begin{teor}
(Teorema de la base de Hilbert) Todo ideal de $k[x_1,\dots,x_n]$ está finitamente generado. 
\end{teor}

\underline{\textit{Demostración.}}
\vspace{2mm}

Observemos que $k[x_1,\dots,x_n]=k[x_1][x_2]\cdots [x_n]$. Sabemos que $k$ es Noetheriano (todos sus ideales son finitamente generados), porque los únicos ideales de $k$ son $\{ 0 \} = <0>$ y $k=<1> $.

\begin{nota}
Si $K$ es cuerpo, $<0>$ y $<1>$ son sus únicos ideales. 
\end{nota}

Vamos a demostrar que si $R$ es un anillo Noetheriano, entonces $R[x]$ es un anillo Noetheriano (y esto termina la demostración por inducción en $n$).

Sea $I$ un ideal de $R[x]$. Vamos a demostrar que está finitamente generado.

\begin{nota}
Dado un polinomio: $p(x)=a_mx^m+\dots + a_1x+a_0$ el elemento $a_m\in R$ se llama \textbf{coeficiente lider}.
\end{nota}

Definimos el ideal $I_L=\{ a \in R : a \text{ es coeficiente lider de algún } p(x)\in I\}$. $I_L$ es un ideal de $R$ por la proposición 3 tel tema 2. Por hipótesis, $I_L$ está finitamente generado, es decir,  $I_L=<c_1,\dots , c_m>\subset R$ para ciertos $c_i,i=1,\dots , m$.

\vspace{2mm}
Tomemos polinomios $g_1,\dots, g_m$ que tengan como coeficientes líderes aquellos $c_i$. Sea $N$ el máximo grado de $g_1,\dots, g_m$.

\vspace{2mm}
Fijado un grado $d$, definimos el ideal
$$I_d=\{a\in R : a \text{ es coeficiente lider de un }f(x)\in R \text{ de grado } d \}\cup \{0\}.$$
$I_d$ es un ideal de $R$, por lo tanto, $I_d=<c_{d,1},\dots , c_{d,m_d}>$. Tomamos $g_{d,1},\dots , g_{d,m_d}$ polinomios de $I$, de grado $d$ que los tengan como coeficientes líderes.
\vspace{2mm}

Veamos que
$$I=<g_1,\dots , g_m, g_{N-1,1},\dots , g_{N-1,m_{N-1}},\dots, g_{1,1},\dots , g_{1,m_1},\dots , g_{0,1},\dots , g_{0,m_0}>$$

Para ello, tomemos un polinomio cualquiera $f\in I$, y vamos a escribirlo como suma de múltiplos de los $g_i,g_{d,j}$. Iremos bajando el grado de $f$, sumándole múltiplos de los $g_i,g_{d,j}$, hasta llegar al polinomio nulo.

Si $deg(f)\ge N \Rightarrow f=a_rx^r+\dots + a_1x+a_0 \Rightarrow a_r\in I_L=<c_1,\dots, c_m>$.

Entonces, $\exists \alpha_1,\dots, \alpha_n\in R$ tales que $a_r=\alpha_1 c_1+\dots + \alpha_m c_m$. Consideramos $f-\alpha_1 x^{r-deg(g_1)}g_1-\dots - \alpha_m x^{r-deg(g_m)}g_m$, así el término de grado $r$ tiene coeficiente $0$. Y así hemos bajado el grado de $f$.

Se itera el razonamiento hasta que $deg(f)<N$. Ahora, se procede a cancelar el término lider restando múltiplos de $g_{d,1},\dots, g_{d,m_d}$, iterando hasta tener el grado $0$, es decir, hasta que el polinomio sea constante, y sea una suma de múltiplos de$g_{0,1},\dots, g_{0,m}$.

Por lo tanto, el polinomio $f$ es suma de múltiplos de los $g_i$ y los $g_{d,j}$. 
\qed

\end{document}