\chapter{Variables continuas}
\section{Distribución uniforme}

Una distribución uniforme en el intervalo $(a,b)$ tiene
como \textbf{función de densidad}
$$f(x)=\frac{1}{b-a}I_{[a,b]}(x),x\in \mathbb{R} $$

Se trata de la selección al azar de un punto del intervalo.
Tiene como \textbf{función de distribución}

$$
F(x)=
\left\{
  \begin{array}{l}
    0, \quad x<a \\
    \frac{x-a}{b-a}, \quad a<x<b \\
    1, \quad x>b
  \end{array}
    \right.
    $$

\begin{itemize*}
\item $E(X)=\frac{a+b}{2}$
  \vspace{2mm}
\item $Var(X)= \frac{(b-a)^2}{12}$
  \vspace{2mm}
\item $E(X^k)=\frac{b^{k+1}-a^{k+1}}{(b-a)(k+1)}$
\end{itemize*}

\section{Distribución exponencial}

Una distribución exponencial de parámetro $\lambda $ tiene
como \textbf{función de densidad}

$$f(x)=\lambda e^{-\lambda x}I_{(0,+\infty)}(x)$$

y como \textbf{función de distribución}

$$F(x)=(1-e^{-\lambda x})I_{(0,+\infty)}(x)$$

\begin{itemize*}
\item $E(X)= \frac{1}{\lambda }$
\item $Var(X)= \frac{1}{\lambda^2 }$
\item $E(X^k)=\frac{k!}{\lambda^k}$
\item No es reproductiva
\end{itemize*}

\section{Distribución Gamma}
$$f(X)=\frac{a^p}{\Gamma (p)}e^{-ax}x^{p-1}I_{(0,+\infty)}$$

\begin{itemize*}
  \item $\Gamma (1) = 1$
  \item $\Gamma (p) = (p-1)\Gamma(p-1)$ 
\end{itemize*}

\section{Distribución normal univariante}

Sea $X$ una variable aleatoria que se distribuye según una
$N(\mu,\sigma^2)$, su función de densidad es de la forma

$$f(x)=\frac{1}{\sigma\sqrt{2\pi}}e^{-\frac{(x-\mu)^2}{2\sigma²}}$$