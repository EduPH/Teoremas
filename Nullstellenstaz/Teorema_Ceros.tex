\documentclass{article}
\usepackage[spanish]{babel}
\usepackage[utf8]{inputenc}
\usepackage[T1]{fontenc}
\usepackage{graphicx}
\usepackage{fancyvrb}              % Verbatim extendido
\usepackage{makeidx}               % Índice
\usepackage{amsmath}               % AMS LaTeX
\usepackage{amsthm} 
\usepackage{latexsym}
\usepackage{vmargin}
\newtheorem{teor}{Teorema}
\newtheorem{nota}{Nota}
\newtheorem{Def}{Definición}
\newenvironment{itemize*}%
  {\vspace*{-0mm}
   \begin{itemize}%
    \setlength{\itemsep}{0pt}%
    \setlength{\parskip}{0pt}}%
  {\vspace*{-0mm}
   \end{itemize}}
\newtheorem{Lem}{Lema}
\setpapersize{A4}
\setmargins{2.5cm}       % margen izquierdo
{1.5cm}                        % margen superior
{16.5cm}                      % anchura del texto
{23.42cm}                    % altura del texto
{10pt}                           % altura de los encabezados
{1cm}                           % espacio entre el texto y los encabezados
{0pt}                             % altura del pie de página
{2cm}                           % espacio entre el texto y el pie de página
\begin{document}

\title{Teorema de los ceros de Hilbert}
\maketitle

\large

\begin{Def}
Un cuerpo $K$ es algebraicamente cerrado si todo polinomio de $K[x]$ tiene una raíz en $K$.  
\end{Def}
\begin{Lem}
Si $K$ y $L$ son cuerpos con $K\subset L$ y $L$ una $K-$ álgebra finitamente generada ($L=K[\alpha_1,\dots, \alpha_r]$). Entonces $L$ es una extensión algebraica de $K$, es decir, todo elemento de $L$ es raíz de algún polinomio de $K[x]$. 
\end{Lem}

\begin{teor}
 \textbf{(Versión débil del Nullstellensatz)} Si $K$ es algebraicamente cerrado e $I$ es un ideal de $K[x_1,\dots,x_n]$. Se tiene que $I\neq k[x_1,\dots, x_n] \Leftrightarrow V(I)\neq \emptyset$.
\end{teor}

\underline{\textit{Demostración:}}
\vspace{3mm}


\framebox{$\Leftarrow$} Trivial, si $I=K[x_1,\dots, x_n]$ entonces $1\in I$, lo que implica que $V(I)=\emptyset$.

\framebox{$\Rightarrow$} Como $I\neq K[x_1,\dots,x_n]$, entonces se tiene que existe $m$ ideal maximal que lo contiene.

Consideremos la proyección canónica:
$$ \pi: K[x_1,\dots,x_n] \rightarrow k[x_1,\dots,x_n]/m$$
Como $m$ es maximal, $K[x_1,\dots,x_n]/m$ es un cuerpo, que denotamos por $L$. $L$ es una $K-$álgebra finitamente generada por $(x_1+m,\dots, x_n+m)$. Aplicando el lema de Zariski se tiene que $L$ es una extensión algebraica de, y como $K$ es algebraicamente cerrado, $L=K$. Por tanto,
$$\pi : k[x_1,\dots,x_n] \rightarrow K$$
Vamos a encontrar un punto $P\in V(I)$. Concretamente, tomamos $P=(\pi(x_1),\dots, \pi(x_n)) \in k^n$. Veamos que $P$ anula a todos los polinomios de $I$.

Dado $f\in I \Rightarrow f\in m \Rightarrow \pi(f)=0 \Rightarrow f(\pi(x_1),\dots, \pi(x_n))=0 \Rightarrow f(P)=0$ \qed

\begin{nota}
  Si $V_1,\dots, V_k$ son variedades afines, entonces $V_1\cup \dots \cup V_k$ es una variedad afin es una variedad afin.
  \begin{itemize*}
  \item Si $\{ V_i \}_{i\in I}$ es un conjunto de variedad afines, entonces $\cap_{i\in I} V_i$ es una variedad afín.
  \item $\emptyset $ es una variedad afín $(V([k[x_1,\dots,x_n])=\emptyset)$.
  \item $k^n$ es una variedad afín $(V(\{0\})=k^n)$.
  \end{itemize*}
\end{nota}
Las variedades afines son los cerrados de una topología de $k^n$, que se llama topología de Zariski.

\begin{Def}
  Dado $I$, un ideal de $k[x_1,\dots, x_n]$, se define el radical de $I$ como:
  $$\sqrt{I}=\{ f \in k[x_1,\dots,x_n] : f^n \in I \text{ para algún } m\ge 1\}$$
  Se dice que un ideal $I$ es radical, si $I=\sqrt{I}$.
\end{Def}

\begin{nota}
  $I\subset \sqrt{I}$, y $\sqrt{I}$ es un ideal.
\end{nota}

\begin{teor}
  \textbf{(Nullstellensatz fuerte)} Si $K$ es algebraicamente cerrado.
  \begin{enumerate}
  \item $\forall I$ ideal de $K[x_1,\dots, x_n]$, $I(V(I))=\sqrt{I}$.
  \item Si $A$ es un subconjunto de $K^n$, $V(I(A))=\bar{A}$. 
  \end{enumerate}
\end{teor}

\textit{\underline{Demostración}}

\vspace{3mm}
\begin{enumerate}
\item \framebox{$\supseteq $} Si $f\in \sqrt{I}$, entonces $f^m\in I \Rightarrow f^m(P)=0, \forall P \in V(I) \Rightarrow f(P)=0, \forall P\in V(I) \Rightarrow f\in I(V(I)). $

\framebox{$\subseteq $} Sea $f\in I(V(I))$. Escribamos $I=<g_1,\dots,g_r>$. Vamos a trabajar en el anillo $K[x_1,\dots, x_n,y]$,y consideremos el ideal:
$$I'=<g_1,\dots,g_r,1-yf>$$
Se oberva que $V(I')=\emptyset$, porque si un punto anula a $g_1,\dots,g_r$, entonces está en $V(I)$, por lo tanto, también anula a $f$ y no anula a $1-yf$.
Por el Nullstellensatz débil, $I'=k[x_1,\dots,x_n,y] \Rightarrow 1\in I' \Rightarrow \exists h_1,\dots,h_r,h_{r+1}\in K[x_1,\dots,x_n,y]$ tales que
$$1=h_1g_1+\dots+h_rh_r+h_{r+1}(1-yf) $$
Sustituyendo $y$ por $\frac{1}{f}$, y quitando denominadores, queda:
$$f^m=f_1g_1+\dots+f_rg_r\in I$$
Luego $f\in \sqrt{I}$. \qed

\item Como $V(I(A))$ es cerrado en la topología de Zariski, (es decir, es una variedad afín) sólo tenemos que demostrar que es el menor cerrado que contiene a $A$. 

Sea $W$ un cerrado que contiene a $A$. $W=V(J)$ para un ideal $J$. 
\begin{equation}
  \begin{split}
    V(J)&\supset A \\
    I(V(J))&\subset I(A) \\
    J\subset \sqrt{J} &\subset I(A) \\
    W=V(J) &\supset V(I(A))
  \end{split}
\end{equation}
\qed
\end{enumerate}

\begin{nota}
El anillo cociente $k[x_1,\dots,x_n]/I$ es un cuerpo si y sólo si $I$ es maximal. 
\end{nota}
\begin{nota}
Los puntos son los conjuntos algebraicos más pequeños, no vacíos. Por lo tanto, se corresponden con los ideales maximales. 
\end{nota}


\underline{\textit{Demostración:}}

\vspace{3mm}

Sea $p=(a_1,\dots,a_n)\in K^n$, veamos que le corresponde el ideal $<x_1-a_1,\dots,x_n-a_n>$, es decir,
$$ I(\{a_1,\dots,a_n\})=\underbrace{<x_1-a_1,\dots,x_n-a_n>}_I$$
El ideal $I$ se anula en el punto $p$, luego se tiene la contención \framebox{$\supseteq $}. Si demostramos que $I$ es maximal, habremos probado la igualdad. Para verlo, definimos el siguiente morfismo de anillos:
\begin{equation}
  \begin{split}
    \varphi: K[x_1,\dots,x_n] &\rightarrow K  \\
    f(x_1,\dots,x_n) & \mapsto f(a_1,\dots,a_n)
  \end{split}
\end{equation}
Se tiene que $Im(\varphi)=K$, pues si $f$ es un polinomio constante, $f(\alpha)=\alpha$. Veamos que el núcleo es $I=<x_1-a_1>$. 
$$ker(\varphi) = <x_1-a_1,\dots,x_n-a_n> $$

\framebox{$\supseteq $} Trivial.

\framebox{$\subseteq $} Sea $f\in ker(\varphi)$, entonces $f(a_1,\dots,a_n)=0$. Podemos escribir cualquier polinomio $f(x_1,\dots,x_n)$ de la siguiente manera:
$$f(x_1,\dots,x_n)=h_1(x_1,\dots,x_n)(x_1-a_1)+h_2(x_2,\dots,x_n)(x_2-a_2)+\dots +h_n(x_n)(x_n-a_n)+\underbrace{\alpha}_{\in K}$$
Entonces, $f(a_1,\dots,a_n)=\alpha$, pero si $f\in ker(\varphi) \Rightarrow \alpha=0$. Por lo tanto, $f\in I$, lo que implica que $\frac{K[x_1,\dots,x_n]}{ker(\varphi)} \simeq Im(\varphi)$.
Tomamos $\frac{K[x_1,\dots,x_n]}{I} \simeq K$, y como $K$ es cuerpo, $I$ es ideal maximal. \qed



\begin{thebibliography}{X}
\bibitem{COX} \textsc{Cox,D.,Little,J.,O'Shea,D.} ,
\textit{Ideals,Varieties,and Algorithms}, Springer-Verlag,2007.
\end{thebibliography}
\end{document}