\chapter{Conjuntos algebraicos afines}

\section{Preliminares algebraicos}
\index{\texttt{anillo}}
\begin{Def}
Un \textbf{anillo} es un conjunto $A$ dotado de dos operaciones binarias, que llamaremos suma (+)  y producto (·) que verifican:

\begin{enumerate}
\item $(A,+)$ Grupo abeliano.
\item Propiedad asociativa del producto:  $a(bc)=(ab)c$
\item Propiedad distributiva del producto respecto de la suma: 
$a(b+c)=ab+ac$
\item Existencia de elemento neutro para el producto: $1_A$ tal que $1_A$·$a=a, \forall a,b,c \in A$
\end{enumerate}
\end{Def}
Algunos ejemplos de anillos:

\begin{itemize*}
\item $A=\{ 0 \}$ es un anillo.
\item $\mathbb{Z}$ el anillo de los enteros.
\item $k[x]$ el anillo de los polinomios con coeficiente en un cuerpo $k$.
\item $A[x]$ el anillo de los polinomios con coeficientes en un anillo $A$.
\end{itemize*}

\index{\texttt{Ideal}}
\begin{Def}
Un subconjunto $I\subseteq A$ de un anillo $A$ se llama \textbf{ideal} si $(I,+)$ subgrupo de $(A,+)$ y $\forall a\in A, \forall b \in I$ entonces $ab\in I$
\end{Def}


\index{\texttt{anillo cociente}}
\begin{Def}
Sea $I\subseteq A$ un ideal, se llama \textbf{anillo cociente} de $A$ por $I$ al conjunto cociente $A/I$ dado por el cociente de $A$ respecto de la siguiente relación de equivalencia en $A$: $a,b\in A$, $a \sim b \Leftrightarrow a-b \in I$
\end{Def}

Denotaremos $[a]= \bar{a} = a+I$ a la clase de equivalencia de $a\in A$ respecto de la relación de equivalencia anterior.

$A/ I$ es anillo con las operaciones + y · naturales:

\begin{itemize*}
\item $\bar{a} + \bar{b} = \bar{a+b}$
\item $\bar{a}$ · $\bar{b} = \bar{ab}$, $\forall \bar{a},\bar{b}\in A \backslash I$
\end{itemize*}

Más ejemplos de anillos:

\begin{itemize*}
\item Sea $m\in \mathbb{Z}, <m>=\{am : a\in \mathbb{Z} \} \subseteq \mathbb{Z}$ ideal. $\mathbb{Z} / m\mathbb{Z}$ anillo cociente.
\end{itemize*}


\index{\texttt{Divisor de cero}}
\begin{Def}
Un elemento $a\in A$ se llama \textbf{divisor de cero} si $\exists b \neq 0, b \in A$ tal que $ab=0$.
\end{Def}


\index{\texttt{Cuerpo}}
\begin{Def}
Un anillo $A \neq 0$ es un \textbf{dominio de integridad} si no tiene divisores de cero distintos del cero. Un dominio de integridad en el que todo elemento no nulo tiene inverso se llama \textbf{cuerpo}.
\end{Def}

\begin{Def}
Un ideal $I\subseteq A$ es \textbf{propio} si $I\neq A$ .
\end{Def}

\begin{Def}
Un ideal $I\subseteq A$ es \textbf{primo} si es propio y $\forall a,b\in A$, si $ab\in I \Rightarrow a\in I$ o $b\in I$.
\end{Def}

\begin{Def}
Un ideal $I\subseteq A$ es \textbf{maximal} si es propio y $\not\exists J$ ideal propio tal que $I \subseteq J, I\neq J$.
\end{Def}


\begin{Prop}
Sea $I\subseteq A$ ideal, entonces:

\begin{enumerate}
\item $I$ primo $\Leftrightarrow A/I$ es dominio de integridad.
\item $I$ maximal $\Leftrightarrow A/I$ es cuerpo. 
\end{enumerate}
\end{Prop}

Ejemplo:

\begin{itemize*}
\item $\mathbb{Z}/m\mathbb{Z}$ es dominio de integridad 
$\Leftrightarrow m =0$ y $m$ primo. ($<m>$ primo o cero).
\item $\mathbb{Z}/m\mathbb{Z}$ es cuerpo $\Leftrightarrow m $ primo. ($<m>$ maximal)
\end{itemize*}


\begin{Def}
Sea $a\in A$, se dice que $a$ es una \textbf{unidad} si $\exists b \in A$ tal que $ab=1_A$ y notaremos $a^{-1}=b$
\end{Def}


\index{\texttt{Irreducible}}
\begin{Def}
Un elemento $a\in A$ se llama \textbf{irreducible} si no es 0 ni unidad y $\forall b,c \in A$ tal que $a=bc$ $\Rightarrow b$ unidad o $c$ unidad.
\end{Def}


\index{\texttt{Dominio de factorización única}}
\begin{Def}
Un \textbf{dominio de factorización única} es un dominio de integridad tal que todo elemento no nulo de $A$ se puede escribir
como producto de elementos irreducibles y unidades. Además, esta factorización es única, salvo producto 
por unidades y el orden de los factores.
\end{Def}

Ejemplos:

\begin{itemize*}
\item $\mathbb{Z}$, $k[x]$
\end{itemize*}


\index{\texttt{Cuerpo de fracciones}}
\begin{Def}
Sea $A$ un dominio de integridad, se llama \textbf{cuerpo de fracciones} de $A$ al conjunto cociente de $A \times (A \backslash \{0 \})$ respecto de la siguiente relación de equivalencia:

\begin{itemize*}
\item $(a,b),(c,d)\in A \times (A\backslash \{ 0 \})$,
  $(a,b) \sim (c,d) \Leftrightarrow ad=bd$ en $A$.
\end{itemize*}
\end{Def}

Denotaremos $\frac{a}{b}$ a la clase de equivalencia de $(a,b)$.

Denotaremos $Q(A)=\frac{ A\times (A \backslash \{ 0 \})}{\sim}$

\begin{Prop}
$Q(A)$ con las operaciones + y · siguientes es un cuerpo:

\begin{itemize*}
\item $\frac{a}{b} + \frac{c}{d}= \frac{ad+bc}{bd}$
\item $\frac{a}{b} \frac{c}{d} = \frac{ac}{bd}$
\end{itemize*}
\end{Prop}
Ejemplo:

\begin{itemize*}
\item $A=\mathbb{Z} \Rightarrow Q(\mathbb{Z})=\mathbb{Q}$
\item $A=k[x] \Rightarrow Q(k[x])=k(x)= \{\frac{f}{g} : f,g\in k[x], g\neq 0  \}$
\end{itemize*}

\begin{Lem}
\textbf{Lema de Gauss} Sea $A$ un dominio de factorización única y $f,g \in A[x]$. Entonces, $c(fg)=c(f)c(g)$.
\end{Lem}


\index{\texttt{Contenido}}
\begin{Def}
El \textbf{contenido} de $f$ , $c(f)$, es el máximo comun divisor de los coeficientes de $f$.
\end{Def}

\begin{Cor}
Si $A$ es dominio de factorización única y $f\in A[x]$ entonces $f$ es irreducible en $A[x] \Leftrightarrow f$ es irreducible en $K[x]$ , donde $K=Q(A)$.
\end{Cor}

\index{\texttt{Subanillo}}
\begin{Def}
Sea $A$ anillo y $B\subseteq A$, diremos que $B$ es \textbf{subanillo} de $A$ si $(B,+)$ es subgrupo abeliano de $(A,+)$,
$1_A \in B$, y $B$ es cerrado para el producto ($ab\in B, \forall a,b\in B$).
\end{Def}


\index{\texttt{Ideal generado por un conjunto}}
\begin{Def}
Un ideal $I\subseteq A$ se dice que está \textbf{generado por un conjunto} 
$ S \subseteq A$ si $I = <S> := \{ \sum_{i=1}^{r}a_is_i : s_i \in S, a_i  \in A, r\in \mathbb{N},i=1,\dots , r \} $
\end{Def}

\index{\texttt{Ideal principal}}
\begin{Def}
Un ideal es \textbf{principal} si está generado por un único elemento.
\end{Def}


\index{\texttt{Dominio de ideales principales}}
\begin{Def}
Un dominio de integridad en el que todo ideal es principal se llama \textbf{dominio de ideales principales}.
\end{Def}

Ejemplos:
\begin{itemize*}
\item $\mathbb{Z},k[x]$ con $k$ cuerpo.
\item $k[x_1, \dots, x_n]$ $(n \le 2)$ es dominio de factorización única, pero no es dominio de ideales principales.
\end{itemize*}

\begin{nota}
Si $A$ es dominio de ideales principales, entonces $A$ es dominio de factorización única.
\end{nota}


\index{\texttt{Morfismo de anillos}}
\begin{Def}
Sea $f: A \rightarrow B$ una aplicación entre anillos, se dice que $f$ es un \textbf{morfismo de anillos} si:

\begin{enumerate}
\item $f(a+b) = f(a)+f(b), \forall a,b \in A$ ($f$ morfismo de grupos abelianos).
\item $f(ab)=f(a)f(b), \forall a,b \in A$.
\item $f(1_A)=1_B$.
\end{enumerate}
\end{Def}

Ejemplo:

\begin{itemize*}
\item id:$A\rightarrow A$ es un morfismo de anillos.
\item $I \subseteq A$ ideal, $\pi : A \rightarrow A/ I$, $a\rightarrow \bar{a}$ es un morfismo de anillos.
\end{itemize*}

\begin{Prop}
Si $f:A \rightarrow B$ es morfismo de anillos y $J\subseteq B$ es un ideal $\Rightarrow $ $f^{-1}(J)$ es un ideal de $A$.
\end{Prop}

En particular, el núcleo de $f$, $ker(f):=f^{-1}(0)$ es un ideal.

\begin{nota}
Un homomorfismo es inyectivo si y sólo si el núcleo es 0.
\end{nota}

\subsubsection{Operaciones con ideales:}

\begin{itemize*}
\item La intersección de ideales es un ideal.
\item La unión de ideales \textbf{NO} es en general un ideal. La unión infinita de ideales encajados sí es un ideal.
\item La suma de ideales es el ideal generado por la unión.
\item El producto de dos ideales $I,J\subseteq A$ es el ideal $$IJ := <\{ab : a\in I, b \in J \} > = \{\sum_{i=1}^{r}a_i b_i : a_i \in I, b_j \in J, r \in \mathbb{N}, i,j=1, \dots, r  \}$$
\end{itemize*}

\textbf{Cuestión(Entrega antes de la clase del viernes 30 Sept.):} Sea $k$ un cuerpo infinito, $F\in k[x_1, \dots , x_n]$. Supongamos que $F(a_1, \dots , a_n ) = 0, \forall (a_1, \dots , a_m) \in k^n$. Hay que probar que $F=0$.


\textbf{Cuestión:}  Sea $k$ un cuerpo, $F\in k[x_1, \dots , x_n], a_1, \dots , a_n \in k$. Probar:

\begin{enumerate}
\item $F=\sum \lambda_{(i)}(x_1-a_1)^{i_1}\dots(x_n-a_n)^{i_n}$ 
\item Si $F(a_1,\dots,a_n) = 0$, probar $F=\sum_{i=1}^n(x_i-a_i)G_i$ con $G_i \in k[x_1,\dots ,x_n]$
\end{enumerate}


Problemas 1.4 y 1.7 Fulton.

\section{El espacio afín y los conjuntos algebraicos}

Sea $k$ cuerpo, se denotará $\mathbb{A}^n=\mathbb{A}^n(k)$ al espacio afín de dimensión $n$ sobre el cuerpo $k$, cuyos elementos son del tipo $(a_1,\dots,a_n)$, $a_i\in k$.


\index{\texttt{Conjunto algebraico afín}}
\begin{Def}
Un \textbf{conjunto algebraico afín} es el conjunto de puntos del espacio afín, $\mathbb{A}^n$, que es solución de un sistema de ecuaciones polinómicas. Es decir, es un conjunto de la forma:

$$ V(S):= \{ p=(a_1,\dots , a_n) \in \mathbb{A}^n | f(a_1,\dots , a_n) = 0, \forall f\in S \} $$

Para cierto $S\subseteq k[x_1,\dots, x_n]$
\end{Def}

Ejemplo:

\begin{itemize*}
\item $S= \{ 0 \} \Rightarrow V(S)= \mathbb{A}^n$ es un conjunto algebraico.
\item $S=k[x_1, \dots, x_n] \Rightarrow V(S)= 0$
\item $S$ es un conjunto finito de polinomios de grado 1 $\Rightarrow V(S)$ variedad lineal afín.
\item $S=\{f \}$ con grado($f$) = 2 $\Rightarrow V(S)$ es una hipercuádrica lugar afín.
\end{itemize*}

\begin{nota}
Si $S=\{f \}, f\neq 0$ diremos que $V(S)$ es una hipersuperficie. En particular, si $n=2 \rightarrow V(f)$ se llama una curva plana afín. 
\end{nota}

\begin{nota}
$S\subseteq k[x_1,\dots x_n]$, $V(S)=\bigcap_{f\in S}V(f)$
\end{nota}

\subsubsection{Propiedades}

\begin{enumerate}
\item $I=<S>\subseteq k[x_1,\dots , x_n] \Rightarrow V(I)=V(S)$. Luego, todo conjunto algebraico es de la forma $V(I)$ para algún ideal.

\item $\{ I_\alpha \}_{\alpha \in A}$ familia de ideales $\Rightarrow V(\bigcup_{\alpha \in A}I_\alpha )=\bigcap_{\alpha \in A} V(I_\alpha)$; En particular, $V(<\cup_\alpha I_\alpha >)=\cap_\alpha V(I_\alpha)$.
\item $I\subseteq J \Rightarrow V(J)\subseteq V(I)$
\item $V(fg)=V(f)\cup V(g)$. Sean $I,J\subseteq k[x_1,\dots,x_n]$ ideales $\Rightarrow V(IJ)=V(I\cap J)=V(I)\cup V(J)$
\item $V(0)=\mathbb{A}^n, V(k[\vec{x}])=\emptyset$
\item Si $p=(a_1,\dots, a_n)\in \mathbb{A}^n, V(x_1-a_1,\dots,x_n-a_n)=\{p \}$.
\end{enumerate}


\begin{Dem}

\framebox{3}. Es obvio pues en $J$ hay más polinomios que en $I$ y, por tanto, todo punto solución de $\{ f=0 : f\in J \}$ es también solución de $\{ f = 0: f \in I \}$. \qed

\framebox{1}. $S\subseteq I \Rightarrow V(I) \subseteq V(S)$

\framebox{$\supseteq $} Sea $p\in V(S) \Rightarrow f(p) = 0, \forall f\in S$. Hay que probar que $f(p)=0, \forall f \in I$.
Sea $f\in I=<S> \Rightarrow \exists g_1,\dots, g_r \in S, \exists h_1,\dots , h_r \in k[\vec{x}] | f=\sum h_i g_i \Rightarrow f(p)=\sum h_i(p)g_i(p) = 0 \Rightarrow p\in V(I)$ \qed

\framebox{4}. $(IJ \subseteq I) \wedge (IJ\subseteq J) \Rightarrow (V(I)\subseteq V(IJ))\wedge (V(J)\subseteq V(IJ)) \Rightarrow V(I)\cup V(J) \subseteq V(IJ)$.

\framebox{$\supseteq$} $P\in V(IJ)$. Supongo $P\notin V(I)$, hay que probar que $P\in V(I)\cup V(J)$. $\exists f\in I$ tal que $f(P)\neq 0$. Sea $g\in J$ cualquiera, entonces $fg \in IJ \Rightarrow (fg)(P)=f(P)g(P)=0 \Rightarrow g(P) =0 \Rightarrow P\in V(J)$ \qed 
\end{Dem}

\begin{Cor}
$\{ V(I) : I\subseteq k[x_1,\dots ,x_n] \text { ideal } \}$ es la familia de cerrados de una topología en $\mathbb{A}^n$, que llamaremos topología de Zariski en $\mathbb{A}^n$.
\end{Cor}

\textbf{Ejercicio 1.3. } Sea $R$ un dominio de ideales principales, sea $p$ ideal primo, $0\neq p \subset R$.
\begin{enumerate}
\item Probar que $p$ está generado por un elemento irreducible.
\item Probar que $p$ es maximal.
\end{enumerate} 

\textit{Solución:}
\begin{enumerate}
\item $\exists a \in R$ tal que $p=<a>$ por ser $R$ DIP. Reducción al absurdo: $a=bc$ con $b,c$ no unidades $\Rightarrow b\in p \vee c \in p$.

Si $b\in p = <a> \Rightarrow \exists d \in R$ tal que $b=da=dbc \Rightarrow dc=1$. Luego, $d$ y $c$ unidades $\rightarrow \leftarrow $

\item Reducción al absurdo. Si $p$ no es maximal $\Rightarrow \exists q$ ideal, $p\subset q \subset R$, $q=<b>$. $b$ unidad ya que $q\neq R$. $a\in <b> \Rightarrow a=cb$ para cierto $c\in R$, $\Rightarrow c$ unidad ($a$ irreducible) $\Rightarrow b=c^{-1}a$ y $p=<a>=<b>=q$
\end{enumerate}

\textbf{Ejercicio 1.5. } Sea $k$ un cuerpo cualquiera. Probar que existe un número infinito de polinomios mónicos irreducibles en $k[x]$. 

\underline{\textit{Solución: }}

Sean $F_1,\dots ,F_n$ todos los polinomios mónicos irreducibles de $k[x]$. Sea $F=1+F_1\cdots F_n$, como $F_i|\prod F_j$ y $F_i \not | 1 \Rightarrow F_i\not | F \Rightarrow F$ irreducible.

\textbf{Ejercicio 1.6. } Probar que cualquier cuerpo algebraicamente cerrado (Todos los polinomios no constantes tienen al menos una raíz) es infinito.

\underline{\textit{Solución: }}

Si $k$ fuera finito $\Rightarrow \{X-a : a\in k \}$ también sería finito, y entraría en contradicción con el ejercicio anterior.

\textbf{Ejercicio 1.8. } Probar que los conjuntos algebraicos de $\mathbb{A}^1(k)$ son $\mathbb{A}^1$ y los conjuntos finitos de puntos de $\mathbb{A}^1$.

\underline{\textit{Solución: }}

$I\subseteq k[x]$ ideal $\Rightarrow_{(k[x] \text{ DIP })}I=<f>$ para cierto $f\in k[x]$.

$V(f)$ es $\mathbb{A}^1$ si $f=0$, y si $f\neq 0$ entonces serían las raíces del polinomio  que no pueden ser más que el grado del polinomio.

\textbf{Ejercicio 1.9. } Si $k$ cuerpo finito, entonces todo conjunto de $\mathbb{A}^n$ es algebraico.

\underline{\textit{Solución: }}

Como $k$ es finito, entonces $\mathbb{A}^n$ es finito, y entonces todo subconjunto de $\mathbb{A}^n$ es finito. Como de las propiedades (5) y (4), los puntos y la unión de conjuntos algebraicos son algebraicos, entonces todo conjunto finito de puntos es algebraico.

\textbf{Ejercicio 1.10. } Ejemplo de conjunto infinito numerable de conjuntos algebraicos que no sea conjunto algebraico.

\underline{\textit{Solución: }}

En $\mathbb{A}^1(k)$ si $\mathbb{Q}\subseteq k$, $\mathbb{Z}$ es un ejemplo.

\textbf{Ejercicio 1.11 } Probar que los siguientes conjuntos son algebraicos:

\begin{enumerate}
\item $\{ (t,t^2,t³): t\in k \} \subseteq \mathbb{A}^3(k)$
\item $\{ (cos(t),sin(t)): t\in \mathbb{R} \} \subseteq \mathbb{A}²(\mathbb{R}$
\item El conjunto de puntos de $\mathbb{A}^2(\mathbb{R})$ con coordenadas polares $(r,\theta)$ tales que $r=sen(\theta)$.
\end{enumerate}

\underline{\textit{Solución: }}

\begin{enumerate}
\item $V(y-x²,z-x³)$
\item $V(x²+y²-1)$
\item $x=rcos(\theta),y=rsin(\theta)$, entonces $x²+y²-y=0$. (Elevando cuadrado y sumando).
\end{enumerate}

\textbf{Ejercicios 1.14,1.15: Entregar 30 Septiembre}

\section{El ideal de un conjunto de puntos en $\mathbb{A}^n$}


\index{\texttt{Ideal de $X$}}
\begin{Def}
Sea $X\subseteq \mathbb{A}^n$ un conjunto cualquiera, el conjunto $I(X):= \{ f \in k[x_1,\dots ,x_n] : f(p)=0, \forall p \in X \}$ es un ideal de $k[x_1, \dots,x_n]$ que llamaremos \textbf{ideal de $X$}.
\end{Def}
Propiedades:

\begin{enumerate}
\item Si $X\subseteq Y \Rightarrow I(Y)\subseteq I(X)$
\item $I(\emptyset )=k[x_1,\dots ,x_n]$, si $k$ infinito, $I(\mathbb{A}^n)=<0>=0$
\item $I(\{ (a_1, \dots , a_n) \}) = <x_1-a_1,\dots ,x_n-a_n>$
\item $S\subseteq IV(S), \forall S\subseteq k[x] ; \forall X\subseteq \mathbb{A}^n, X\subseteq VI(X)$
\item $IVI(X)=I(X); VIV(S)=V(S)$
\item $I(X)$ es radical
\item $I(X\cup Y)=I(X)\cap I(Y)$
\end{enumerate}

\begin{Dem}
\framebox{1}. $X\subseteq Y$. Sea $f\in I(Y) \Leftrightarrow f(p)=0, \forall p \in Y \Rightarrow f(p)=0, \forall p\in X \Leftrightarrow f\in I(X)$ \qed

\framebox{3}. $S\subseteq IV(S), S\subseteq k[\vec{x}]$. Sea $f\in S$, hay que probar que $f(p)=0, \forall p\in V(S) \Rightarrow $ se tiene. \qed

\framebox{4}. $IVI(X)=I(X)$, por (3), tenemos \framebox{$\supseteq $}, hay que probar \framebox{$\subseteq $}. Sea $f\in IVI(X) \Rightarrow f(p)=0, \forall p \in VI(X) (\supseteq X )  \Rightarrow f\in I(X)$. (La otra igualdad es análoga). \qed

\framebox{5}. $I(X)= \{ f\in k[\vec{x}] : f(p) = 0, \forall p\in X \} \subseteq \sqrt{I(X)}$. Sea $f\in  \sqrt{I(X)} \Rightarrow \exists m \ge 0 $ tal que $f^m\in I(X) \Rightarrow f^m(p)=0, \forall p\in X \Rightarrow f(p)=0, \forall p\in X \equiv f\in I(X) $ \qed
 
\framebox{6}. $I(X\cup Y) = I(X)\cap I(Y)$. $(X\subseteq X\cup Y)\wedge (Y\subseteq X\cup Y) \Rightarrow (I(X\cup Y)\subseteq I(X))\wedge (I(X\cup Y)\subseteq I(Y))$.

Sea $f\in I(X)\cap I(Y) \Rightarrow f(p)=0 \Rightarrow f(p) = 0 \forall p \in X\cup Y$.

\end{Dem}

\begin{Def}
Dado un ideal $I\subseteq A$, se llama \textbf{radical} de $I$ al conjunto $\sqrt{I} := \{a \in A : \exists m \ge 0, a^m \in I \}$. Un ideal $J \subseteq A$ se llama radical si $J= \sqrt{J}$.
\end{Def}

\textbf{Ejercicio 1.16 } $V=W \Leftrightarrow  I(V)=I(W)$

\underline{\textit{Solución: }}

\framebox{$\Leftarrow $} $I(V)=I(W) \Rightarrow V=VI(V)=VI(W)=W$

\textbf{Ejercicio 1.17 } Sea $V\subseteq \mathbb{A}^n$ conjunto algebraico, $p\notin V$. 
\begin{enumerate}

\item Probar que existe un polinomio $F\in k[x_1,\dots, x_n]$ tal que $F(Q)=0, \forall Q\in V$ pero $F(p)=1$.
\item Sean $P_1,\dots , P_r \in \mathbb{A}^n$ distintos, $P_i\notin V$. Probar que existen $F_1,\dots ,F_r \in k[\vec{x}]$ tales que $F_i(P_j) = 0$ si $i\neq j$, y $F_i(P_i)=1$ . 
\item $a_{ij}\in k, i\le i,j \le r$, hay que probar que existe $G_i \in I(V) $ tal que $G_i(P_J)=a_{ij}, \forall i,j$
\end{enumerate}

\underline{\textit{Solución: }}

\begin{enumerate}
\item Sea $G\in I(V)-I(V\cup \{P \}) \Rightarrow G(Q) = 0, \forall Q \in V, G(p)=a \neq 0 \Rightarrow F = \frac{G}{a}$
\item Sea $W_i = V \cup P_1 \cup \cdots \cup P_{i-1}\cup P_{i+1}\cup \cdots  \cup P_r$. $W_i\neq W_i\cup \{ P_i \}$. $I(W_i)\neq I(W_i\cup \{P_i \}) \Rightarrow \exists F_i \in I(V) $ tal que $F_i(P_i)=1, F_i(P_j)=0, \forall j\neq i$.
\item Usando ayuda y apartado anterior.
\end{enumerate}

\textbf{Ejercicio 1.18 } Sea $I\subseteq R$ ideal. Si $a^n\in I, b^m \in I$, hay que probar que $(a+b)^{n+m}\in I$.

\underline{\textit{Solución: }}

$\sum_{k=0}^{n+m}  {n+m \choose k} a^kb^{n+m-k}$, se tiene $k\ge n \Rightarrow a^k \in I$ o bien $n+m -k \ge m \Rightarrow b^{m+n}\in I$.

$\sqrt{I}$ ideal. $0\in \sqrt{I}$. Sean $a,b\in \sqrt{I} \Rightarrow a^n,b^m \in I$, por lo anterior $(a-b)^{n+m}\in I \Rightarrow a-b\in \sqrt{I}$.

$a\in A,b\in \sqrt{I} \Rightarrow a^nb^m\in I \Rightarrow ab\in \sqrt{I}$.

Todo ideal primo es radical. Sea $J$ un ideal primo de $A$, $J\subseteq \sqrt{J}$. Sea $a\in \sqrt{J} \Rightarrow a^m\in J \Rightarrow a\in J \vee a^{m-1}$, y por inducción $a\in J$, luego $J$ es radical.

\textbf{Ejercicio 1.19 } $I=<x²+1> \subseteq \mathbb{R}[x]$ es primo, luego radical. Pero, $I\neq I(X), \forall X\subseteq \mathbb{A}^1(\mathbb{R}).$

\underline{\textit{Solución: }}

$\frac{\mathbb{R}[x]}{<x²+1>} \cong \mathbb{C}$ cuerpo $\Rightarrow I$ maximal $\Rightarrow I$ primo $\Rightarrow I$ radical.

Reducción al absurdo: Si $I=I(X) \Rightarrow V(I)=VI(X) \Rightarrow IV(I) = I(X), I(\emptyset) = \mathbb{R}[x] \neq I$.

\textbf{Cuestiones a entregar 1.20,1.21}

\section{El teorema de la base de Hilbert}

Sea $k$ cuerpo.

\begin{Teo}
Sea $V\subseteq \mathbb{A}^n$ conjunto algebraico cualquiera, entonces $V$ es intersección finita de hipersuperficies.
\end{Teo}

\begin{Dem}
Sabemos que $V=V(I)$ para algún ideal $I\subseteq k[x_1,\dots, x_n]$.  En particular, \framebox{si todo ideal de $k[x_1,\dots,x_n]$ es finitamente generado} $\Rightarrow I=<f_1,\dots,f_r >$ para cierto $f_i\in k[\vec{x}] \Rightarrow V=V(I)=V(f_1,\dots , f_n) = V(f_1) \cap V(f_2)\cap \dots \cap V(f_r)$.\qed 
\end{Dem}

\begin{nota}
El contenido recuadrado en la demostración se demostrará más adelante.
\end{nota}

\begin{Def}
Un anillo $A$ se dice \textbf{Noetheriano} si todo ideal de $A$ es finitamente generado.
\end{Def}

\textbf{Ejemplos:} cuerpos, dominios de ideales principales.

\begin{Teo}
\textbf{de la base de Hilbert}. Si $R$ es un anillo Noetheriano $\Rightarrow R[x]$ es Noetheriano.
\end{Teo} 

\begin{Dem}
Sea $I\subseteq R[x]$ ideal cualquiera, hay que probar que $I$ es finitamente generado. Sea $J= \{ \text{(coeficientes líderes) } cl(f) : f\in I \}$, sean $cl(f), cl(g),f,g \in I$. Sea $m=deg(f), n=deg(g), n\le m, h=x^{m-n}g-f\in I$ y $J \ni cl(h)=cl(g)-cl(f)$. 

$J\subseteq R$ ideal $\Rightarrow_{(R \text{ Noether })}J=<a_1,\dots,a_r>$, $a_i=cl(f_i)$ para cierto $f_i\in I$.

Sea $N=max \{deg(f_i) \}$, para cada $m\le N$, definimos $J_m=\{ cl(f): f\in I \wedge deg(f)\ge m \}$, entonces cada ideal $J_m\subseteq R \Rightarrow \exists \{f_{mj} \}_{j\in \Lambda_m}$ finitos que genera a $J_m$. $f_{mj}\subset I$.

$I'=<f_1,\dots,f_r,f_{mj}, m\ge N, j\in \Lambda_m >$. Veamos que $I=I'$. ($\Lambda_m$ es finito)

$I'\subseteq I$ es obvio porque $f_i,f_{mj}\in I$. La contención contraria se demuestra por reducción al absurdo.
$I'\subset I \Rightarrow$ sea $g\in I \backslash I'$ de grado mínimo.


Si $deg(g)>N$, como $cl(g)\in J \Rightarrow cl(g)=\sum_{i=1}^r b_ia_i, b_i \in \mathbb{R}$, entonces puedo tomar $g-\sum h_if_i$ con $cl(h_i)=b_i$.  \\
$deg(h_i)=\underbrace{deg(g)}_{>N}-\underbrace{deg(f_i)}_{\le N}$. $deg(g-\sum h_if_i) < deg(g) \Rightarrow_{(g \text{ de grado mínimo}} g-\sum h_if_i \in I' \Rightarrow g\in I' \rightarrow \leftarrow $.

Luego $I=I'$ es finitamente generado. \qed

\end{Dem}

\begin{Cor}
$R$ Noetheriano $\Rightarrow R[x_1,\dots,x_n]$ Noetheriano.
\end{Cor}

\begin{Dem}
Por inducción usando el teorema.

\framebox{$n=1$} Se aplica el teorema.

\framebox{$n\rightarrow n+1$} $R[x_1,\dots, x_n] \cong P[x_1,\dots , x_{n-1}][x_n]$
\end{Dem}

\begin{Cor}
Si $k$ es cuerpo $\Rightarrow k[x_1,\dots ,x_n]$ es Noetheriano.
\end{Cor}

\begin{Cor}
$\mathbb{Z}[x_1,\dots , x_n]$ es Noetheriano
\end{Cor}

\textbf{Cuestión a entregar el viernes 7 de octubre:} Problema 1.22.


\section{Componentes irreducibles de un conjunto algebraico}

Sea $V\subseteq \mathbb{A}^n$ un  conjunto algebraico.


\index{\texttt{Conjunto algebraico irreducible}}
\begin{Def}
Se dice que $V$ es \textbf{irreducible} si no existen $V_1,V_2$ conjuntos algebraicos tales que $V=V_1\cup V_2$, $V_i\subset V$ con $V_i \neq V$.
\end{Def}

\begin{Prop}
$V\subseteq \mathbb{A}^n$ es irreducible $\Leftrightarrow I(V)\subset k[x_1,\dots ,x_n]$ es ideal primo.
\end{Prop}

\begin{Dem}
\framebox{$\Rightarrow$}


Lo demostramos por reducción al absurdo. Supongamos que $I(V)$ no es primo. Entonces, $ \exists f,g \in k[\vec{x}]$ tal que $fg \in I(V)$ pero $f\notin I(V) \wedge g\notin I(V)$. $V\subseteq VI(V) \subseteq V(fg) =V(f)\cup V(g)$.

Como $V\subseteq V(f)\cup V(g) \Rightarrow V=V\cap (V(f)\cup V(g)) = [V\cap V(f)]\cup [V\cap V(g)]$, entonces como $V$ es irreducible se tendrá que $V=V\cap V(f)$ o que $V=V\cap V(g)$, lo demostramos para el primer caso, el otro es análogo.

Si $V=V\cap V(f) \Rightarrow V\subseteq V(f) \Rightarrow f \ni IV(f) \subseteq I(V) \rightarrow \leftarrow $

\framebox{$\Leftarrow $}


$\exists V_1,V_2 \subset V, V=V_1\cup V_2 \Rightarrow I(V)\subset_{\text{estrictamente}}I(V_1)\cap I(V_2) \wedge I(V)=I(V_1\cup V_2)=I(V_1)\cap I(V_2) \rightarrow \leftarrow$. \qed
\end{Dem}


\begin{Lem}
Sea $A$ un anillo Noetheriano y sea $S \neq \emptyset$ un conjunto de ideales de $A$. Entonces existe un ideal en $S$ que es maximal en $S$ respecto del orden parcial dado por la inclusión.
\end{Lem}

\begin{Dem}
Se demuestra por reducción al absurdo. Supongamos que $S\neq \emptyset$ no tiene elementos maximales como $f\neq \emptyset \Rightarrow \exists I_1 \in S$ ideal que no es maximal en $S \Rightarrow \exists I_2 \in S, I_1 \subset I_2$. Por inducción se construye una cadena de ideales infinita en $S$: 

$$ I_1 \subset I_2 \subset \dots $$

$\Rightarrow I=\cup_{n\ge 1} I_n \subseteq A$ ideal (Noetheriano) $\Rightarrow I$ es finitamente generado.

Cada $a_i\in I=\cup I_n \Rightarrow \exists m_i\ge 1 $ tal que $a_i \in I_{m_i} \Rightarrow a_1,\dots , a_r \in I_m$ si $m=max (m_i)$.

$I_m\subseteq I_{m+1} \dots \subseteq I$, $I_m =<a_1,\dots , a_r >=I=I_{m+1}=\dots \rightarrow \leftarrow $
\end{Dem}

\begin{Cor}
Todo conjunto $S\neq \emptyset $ de conjuntos algebraicos de $\mathbb{A}^n$ posee un elemento minimal respecto de la inclusión. 
\end{Cor}

\begin{Dem}
Basta aplicar el lema anterior a $S'=\{ I(V): V\in S \}$.
\end{Dem}

\begin{Teo}
Todo conjunto algebraico $V\subseteq \mathbb{A}^n$ se puede escribir de forma única como $V=V_1\cup \dots \cup V_r$, con $V_i$ irreducible, $V_i \not \subset V_j, \forall i\neq j$
\end{Teo}

\begin{Dem}
Se demuestra por reducción al absurdo. Supongamos que existen conjuntos algebraicos que no pueden escribirse de esta forma. 

$S= \{ V \subseteq \mathbb{A}^n \text{ conjunto algebraico que no se escriben de esta forma} \} \neq \emptyset \Rightarrow \exists V\in S$ minimal respecto de la inclusión.

En particular, $V\in S \Rightarrow V$ reducible $\Rightarrow \exists V_1 , V_2 \subset V$ tales que $V=V_1\cup V_2$, como $V$ es minimal en $S$, eso significa que $V_1,V_2 \notin S \Rightarrow$ 
$$V1=W_1\cup \dots \cup W_r \text{ con } W_i\not \subset W_j, \forall i \neq j$$
$$V=W_1'\cup \dots \cup W_s' \text{ con } W_i'\not \subset W_j', \forall i \neq j$$
Entonces, $V=W_1\cup \dots \cup W_r \cup W_1'\cup \dots \cup W_s'$
Si algún $W_i\subset W_{j(i)}'$ o bien $w_j'\subseteq W_{i(j)}$ se quita el menor. $V\notin S \rightarrow \leftarrow $.

Falta probar la unicidad: $V=W_1\cup \dots \cup W_r =W_1'\cup \dots \cup W_s'$, cada $W_i,W_j'$ irreducible, y tales que $W_i\not \subset W_j, W_i' \not \subset W_j', \forall i \neq j$.  Entonces, como $W_i \subseteq V= \cup_j W_j' \Rightarrow W_i=\cup_{j=1}^s(W_i\cap W_j')$, $W_i$ es irreducible, entonces existe $j(i)$ tal que $W_i=W_i\cap W_{j(i)}' \Rightarrow W_i \subseteq W_{j(i)}'\subseteq W_{i(j(i))}$. Análogamente, $W_j'\subseteq W_{i(j)} \Rightarrow W_i=W_j{j(i)}'$, con lo cual, $r=s$ y $W_i=W_{j(i)}', i=1,\dots r$.
\end{Dem}

\textbf{Problema 1.23: } \textit{Dar un ejemplo de colección $S$ de ideales en un anillo Noetheriano tal que ningún elemento maximal sea ideal maximal.}

\underline{\textit{Solución: }}

Sea el anillo $A=k[x,y]$ con $k$ cuerpo. $S= \{ <x^m >, m\ge 1 \}$, $<x^{m+1}>\subset <x^m>$, pero estrictamente contenido. Entonces, el único elemento maximal de $S$ es $<x>$ que no es ideal maximal ya que $<x>\subset <x,y> \subset A$.

\textbf{Problema 1.24: } \textit{ Todo ideal propio $J$ de un anillo Noetheriano está contenido en un ideal.}

\underline{\textit{Solución: }}

Basta aplicar el lema 1.5.1 a $S=\{ I \subset A \text{ ideal } : J\subseteq I \} $.

\textbf{Cuestión a entregar el viernes 7 de octubre:} Escribir la sección 1.6: Subconjuntos algebraicos en el plano.


\section{Subconjuntos algebraicos en el plano}

Encontremos todos los subconjuntos algebraicos del espacio afín $\mathbb{A}^2(k)$. Por el \textit{teorema 1.5.1} es suficiente encontrar los conjuntos algebraicos irreducibles.

\begin{Prop}
Sean $F$ y $G$ polinomios, ambos en $k[X,Y]$ sin factores comunes. Entonces $V(F,G)=V(F)\cap V(G)$ es un conjunto finito de puntos.
\end{Prop}

\begin{Dem}
  $F$ y $G$ no tienen factores comunes en $k[X][Y]$, así que tampoco tienen factores comunes en $k(X)[Y]$ (Son isomorfos). $k(X)[Y]$ es un dominio de ideales principales (de hecho, fue uno de los ejemplos que dimos en clase), $gcd(F,G)=1$ en $k(X)[Y]$, así que existen $R,S\in k(X)[Y]$ tales que $RF+SG=1$ por el teorema de Bézout. 

  Sea $D\in k[X]$ un polinomio no nulo tal que $DR=A$, $DS=B\in k[X,Y]$. Entonces $AF+BG=D$.

  Si $(a,b)\in V(F,G)$, entonces $D(a)=0$. Pero, $D$ tiene una cantidad finita de raíces. Esto prueba que solo una cantidad finita de coordenadas de $X$ pertenecen a $V(F,G)$.  Análogamente para las coordenadas $Y$ se aplica el mismo razonamiento en $k(Y)[X]$, y eso implica que solo puede haber una cantidad finita de puntos. \qed

\end{Dem}

\begin{Cor}
Si $F$ es un polinomio irreducible en $k[X,Y]$ tal que $V(F)$ es infinito, entonces $I(V(F))=F$, y $V(F)$ es irreducible.
\end{Cor}

\begin{Dem}
  Si $G\in I(V(F))$, $F$ divide a $G$,ésto es claro pues tienen raíces comunes pero $F$ es irreducible. Por lo tanto, se tiene que $G\in <F>$ entonces $V(F,G)$ es infinito ($V(F)\subset V(G) \Rightarrow V(F,G)=V(F)\Rightarrow \text{ infinito}$.Por la proposición anterior  tienen factores comunes, y $I(V(F))\subset <F>$ . Además, $<F> \subset I(V(F))$, y por la \textit{proposición 1.5.1}, $V(F)$ es irreducible. ($<F>$ es ideal primo al ser $F$ irreducible). \qed
\end{Dem}
\begin{nota}
La proposición 1.5.1 dice que $V(F)$ es irreducible si y sólo si $<F>$ es ideal primo.
\end{nota}

\begin{Cor}
Supongamos que $k$ es infinito. Entonces los subconjuntos algebraicos irreducibles de $\mathbb{A}^2(k)$ son: $\mathbb{A}^2(k),\emptyset$, puntos, y curvas planas irreducibles $V(F)$, donde $F$ es un polinomio irreducible y $V(F)$ es infinito.
\end{Cor}

\begin{Dem}
  Sea $V$ un conjunto algebraico irreducible en $\mathbb{A}^2(k)$. Tenemos los siguientes casos:
  \begin{itemize*}
  \item $V$ es finito.
  \item $I(V)= 0$
  \item Si $\exists F$ polinomio no constante. Como $I(V)$ es primo (pues $V$ es irreducible), podemos considerar que $F$ es irreducible, pues en caso contrario podríamos tomar el factor irreducible de $F$ que pertenezca a $I(V)$. $I(V)=<F>$ pues si $G\in I(V)$ y $G\notin <F>$, entonces $V\subset V(F,G)$ que es finito por la proposición 1.6.1.
  \end{itemize*}
\qed
\end{Dem}

\begin{Cor}
Sea $k$ algebraicamente cerrado, $F$ un polinomio no constante en $k[X,Y]$. Sea $F=F_1^{n_1}\cdots F_r^{n_r}$ la descomposición de $F$ en factores irreducibles. Entonces $V(F)=V(F_1)\cup \dots \cup V(F_r)$ es la descomposición de $V(F)$ en componentes irreducibles, y $I(V(F))=<F_1\cdots F_r>$.
\end{Cor}

\begin{Dem}
 $F_i\not| F_j, \forall i\neq j$, así que no hay relaciones de inclusión entre $V(F_i)$. Se tiene que $I(\cup_i V(F_i))=\cap_iI(V(F_i))=\cap_i<F_i>$. Como cualquier polinomio es divisible por cada $F_i$, también lo será por $F_1\cdots F_r$, $\cap_i<F_i>=<F_1\cdots F_r>$. Notar que $V(F_i)$ es infinito para cada $i$, ya que $k$ es algebraicamente cerrado como se demostró en el problema 1.14. \qed
\end{Dem}

\subsection{Problemas}

\textbf{Problema 1.30: } Sea $k=\mathbb{R}$.
\begin{enumerate}
\item Prueba que $I(V(X^2+Y^2+1))=<1>$. \\
  \underline{\textit{Solución: }}

  $X^2+Y^2+1$ es irreducible en $\mathbb{R}$, y además no existe $(a,b)\in \mathbb{A}^2(\mathbb{R})$ tale que $a^2+y^2+1=0 \Rightarrow V(X^2+Y^2+1)=\emptyset.$ Y se tiene $ I(\emptyset) = <1> =k[X,Y]$
\item Prueba que cada subconjunto algebraico de $\mathbb{A}^2(\mathbb{R})$ es igual a $V(F)$ para algún $F\in \mathbb{R}[X,Y]$. \\
  \underline{\textit{Solución: }}

  Los subconjuntos algebraicos irreducibles de $\mathbb{A}^2(\mathbb{R})$ por el \textit{corolario 1.6.2} son: $\mathbb{A}^2, \emptyset$ y curvas planas. Las curvas planas cumplen lo que nos piden. $\mathbb{A}^2=V(\emptyset)$, y $\emptyset = V(k[X,Y])$.

  Los subconjuntos algebraicos no irreducibles se pueden descomponer en unión de subconjuntos algebraicos irreducibles, y por el \textit{corolario 1.6.3}, el producto de los polinomios que generan dichos subconjuntos irreducibles cumple lo pedido.
\end{enumerate}

\begin{nota}
$V(F)=V(F_1,\dots,F_r), F=\sum F_i^2 \text{ si } F=0 \Rightarrow$ todos los $F_i=0$.
\end{nota}

\textbf{Problema 1.31: }
\begin{enumerate}
\item Encuentra los componentes irreducibles de $V(Y^2-XY-X^2Y+X^3)$ en $\mathbb{A}^2(\mathbb{R})$, y también en $\mathbb{A}^2(\mathbb{C})$. \\
  \underline{\textit{Solución: }}

  $Y^2-XY-X^2Y+X^3=(X-Y)(X^2-Y)$ en $\mathbb{A}^2(\mathbb{R}) \Rightarrow V(Y^2-XY-X^2Y+X^3)=V(X-Y)\cup V(X^2-Y)$.

  $Y^2 − XY − X^2Y + X^3= (X-Y)(X+\sqrt{Y})(X-\sqrt{Y})$ en $\mathbb{A}^2(\mathbb{C}) \Rightarrow V(Y^2-XY-X^2Y+X^3)=V(X-Y)\cup V(X-\sqrt{Y})\cup V(X+\sqrt{Y})$.
  
\item Haz lo mismo para $V(Y^2-X(X^2-1))$, y para $V(X^3+X-X^2Y-Y)$.
  \begin{itemize*}
  \item En $\mathbb{A}^2(\mathbb{R})$: \\
    - $Y^2-X(X^2-1)=(Y+\sqrt{X^3+X})(Y-\sqrt{X^3+X}) \Rightarrow V(Y^2-X(X^2-1))=V(Y+\sqrt{X^3+X})\cup V(Y-\sqrt{X^3+X})$ \\
    - $X^3+X-X^2Y-Y= (X^2+1)(X-Y) \Rightarrow V(X^3+X-X^2Y-Y)=V(X^2+1)\cup V(X-Y)$
  \item En $\mathbb{A}^2(\mathbb{C})$: \\
    - $X^3+X-X^2Y-Y= (X-i)(X+i)(X-Y) \Rightarrow V(X³+X-X²Y-Y)=V(X-i)\cup V(X+i) \cup V(X-Y)$ 
  \end{itemize*}
  
\end{enumerate}
\section{El teorema de los ceros (Nullstellensatz) de Hilbert}

Sea $k$ cuerpo algebraicamente cerrado (ejemplo $\mathbb{C}$). 

\begin{Teo}
\textbf{debil de los ceros.} Sea $I\subset k[\vec{x}]$ ideal propio, entonces $V(I)\neq \emptyset $.
\end{Teo}

\begin{Dem}
Existe $m\subset k[\vec{x}]$ ideal maximal tal que $I\subseteq m \Rightarrow V(m)\subseteq V(I)$. Basta probar que $V(m)\neq \emptyset $. Como $m$ es maximal, se tiene que $\frac{k[\vec{x}]}{m}$ es un cuerpo.

$k\subseteq L = \frac{k[\vec{x}]}{m}$ extensión de cuerpos. \framebox{Supongamos que $k=L$} $\Rightarrow \forall i = 1,\dots , n$, se tiene que $\bar{x_i}=\bar{a_i}$ en $L, a\in k$.

$\Rightarrow \underbrace{<x_1-a_1,\dots ,x_n-a_n>}_{\text{maximal}} \subseteq m \Rightarrow m = <x_1-a_1,\dots ,x_n-a_n> \Rightarrow V(m)=V(x_1-a_1,\dots , x_n-a_n)=\{(a_1,\dots ,a_n)\}\neq \emptyset $
\end{Dem}

\begin{nota}
La suposición en el recuadro queda pendiente de probar.
\end{nota}

\begin{Teo}
\textbf{de los ceros.} $k$ algebraicamente cerrado, $I\subseteq k[x_1,\dots ,x_n]$ ideal $\Rightarrow IV(I)=\sqrt{I}$
\end{Teo}

\begin{Dem}
\begin{itemize*}
\item $\sqrt{I}\subseteq I(V(I))$ (ejercicio)
\end{itemize*}
\framebox{$ \subseteq$} Sea $f\in IV(I)$, hay que probar que $\exists m \ge 1$ tal que $f^m\in I$.  Si $I=<f_1,\dots ,f_r >$, $f$ se anula en todo punto que sea cero común de $f_1,\dots ,f_r$. Si consideramos otro ideal $J=<f_1,\dots ,f_r,f \underbrace{x_{n+1}}_h-1>\subseteq k[x_1,\dots ,x_n,x_{n+1}]$ ideal. $\forall P\in V(f_1,\dots ,f_r) \Rightarrow f(P)=0 \Rightarrow h(P) =-1 \neq 0$. Por tanto $V(I)=\emptyset \Rightarrow_{\text{teo. ceros débil}} J=k[x_1,\dots , x_{n+1}] \Leftrightarrow 1 \in J$.

$\Rightarrow 1 = g_1f_1 + \dots + g_rf_r + g_{r+1}(f x_{n+1}-1)$. Podemos hacer el cambio $x_{n+1}=\frac{1}{y}$, y así obtenemos :
$$1= \sum_j g_j(x_1,\dots ,x_n, \frac{1}{y})f_j(x)+g_{r+1}(x_1,\dots ,x_n,\frac{1}{y})(f\frac{1}{y}-1)$$

Tomo $N\ge 1$ suficientemente grande tal que $y^{N-1}g_j(x_1,\dots ,x_n,\frac{1}{y})\in k[x_1,\dots ,x_n,y]$

$$y^N = \sum_j g'_j(\vec{x},y)f_j(\vec{x})+g'_{r+1}(\vec{x},y)(f-y) $$
$\Rightarrow f^N=\sum g_j'(\vec{x},f(\vec{x}))f(x) \in I$ \qed
\end{Dem}

\begin{Cor}
Existe una correspondencia biyectiva entre ideales radicales de $k[\vec{x}]$ y conjuntos algebraicos de $\mathbb{A}^n$.

$$\{ I\subseteq k[\vec{x}] \text{ ideal } : I=\sqrt{I} \} \leftrightarrow \{v\subseteq \mathbb{A}^n \text{ conjunto algebraico } \}$$
$$ I\rightarrow_V V(I)$$
$$I(X) \leftarrow_I X$$
\end{Cor}

\begin{Dem}
$X$ conjunto algebraico $\Rightarrow $ $VI(X)=X$ (propiedad ya vista) $\Rightarrow V\circ I = id$.

$I$ ideal radical $\Rightarrow IV(I)=\sqrt{I}=I \Rightarrow I\circ V =id$.

Entonces se deduce que $I$ y $V$ son biyectivas y una es la inversa de la otra. 
\end{Dem}

\begin{Cor}
Esta correspondencia lleva ideales primos en conjuntos algebraicos irreducibles y viceversa.
\end{Cor}

\begin{Cor}
Esta correspondencia lleva ideales maximales en puntos y viceversa.
\end{Cor}

\begin{Cor}
Esta correspondencia lleva ideales principales en hiperespacios y viceversa.
\end{Cor}

\begin{Cor}
Si $V$ es una hipersuperficie, i.e, $V=V(f)$ para $f\in k[\vec{x}]\backslash k$ y sea $f=\prod_{j=1}^rf_j^{m_j}$ su descomposición en factores irreducibles entonces $V(f) = V(f_1)\cup \dots \cup V(f_r)$ es su descomposición en componentes irreducibles.
\end{Cor}

\begin{Cor}
Sea $k$ algebraicamente cerrado e $I\subseteq k[x_1,\dots,x_n]$ ideal, encontes $V(J)$ finito si y solo si $\frac{k[x_1,\dots,x_n]}{I}$ es un $k$- espacio vectorial de dimensión finita. Además, en tal caso, $\#V(J) \le dim_k( \frac{k[\vec{x}]}{I})$
\end{Cor}

\begin{Dem}
  $\# V(I) \le dim_k ( \frac{k[\vec{x}]}{I})$. Sean $P_1,\dots,P_r \in V(J)$ distintos $\Rightarrow \exists f_1,\dots,f_r \in k[\vec{x}$ tales que
\begin{equation*}
  f_j(P_j)= \left\{\begin{array}{ll}
    1 \text{ si } i=j \\
    0 \text{ si } i\neq j
    \end{array} \right.
\end{equation*}
Veamos que $\bar{f}_1,\dots,\bar{f}_r\in \frac{k[\vec{x}]}{J}$ son $k$- linealmente independientes.

Supongamos que $\exists \lambda_1,\dots, \lambda_r\in k$ tales que $\sum \lambda_i \bar{f}_i =0$ en $\frac{k{\vec{x}}}{J} \Rightarrow \sum \lambda_if_i \in J \Rightarrow \sum_{i=1}^r\lambda_i f_i(P_j)=0, \forall j=1,\dots,r$, luego $\bar{f}_1,\dots, \bar{f}_r$, linealmente independientes en $\frac{k[\vec{x}]}{J}\Rightarrow dim \frac{k[\vec{x}]}{J} \ge r$.

Si $V(J)$ es infinito $\Rightarrow dim \frac{k[\vec{x}]}{J} \ge r, \forall r \in \mathbb{N} \Rightarrow dim \frac{k[\vec{x}]}{J}=\infty$.

Si $V(J)$ tiene exactamente $r$ puntos $\Rightarrow dim \frac{k[\vec{x}]}{J}<\infty \Rightarrow V(J)$ finito.

Supongamos ahora que $V(J)=\{P_1,\dots, P_r \}$ es finito, $P_i=(a_{1i},\dots,a_{ni})$.

Sea $f_j=\prod_{i=1}^r (x_j-a_{ji}) \Rightarrow f_j(P_i)=0, \forall i,j, i=1,\dots, r, j=1,\dots, n \Rightarrow f_j\in IV(J)=_{k\text{ alg.cerrado}}\sqrt{J} \Rightarrow \exists m_j \ge 0$ tal que $k[x_j] \ni f^m_j\in J$, $f_j^{m_j}=x_j^{l_j}+$ términos de menor grado en $x_j \Rightarrow \bar{x_j}^{l_j}$= clase de suma finita de términos de $k[x_j]$ y grado $\le l_j-1  \Rightarrow \{\prod_{j=1}^n x_j^{\bar{k_j}}: 0\le k_j \le l_j,\forall j=1,\dots, n \}$ sistema generador del k-espacio vectorial $\frac{k[\vec{x}]}{J} \Rightarrow \frac{k[\vec{x}]}{J}$ tiene dimensión finita.
\end{Dem}

\textbf{Ejercicio 1.12: } $C\subseteq \mathbb{A}^2$ curva, $L\not \subset C, L\subseteq \mathbb{A}^2$ recta. $C=V(F),f\in k[x,y]$. Hay que probar que $C\cap L$ conjunto finito de cardinal $\le n= deg(f)$.

\underline{\textit{Solución: }}

$L=V(y-(ax+b)) \Rightarrow L\cap C = V(y-(ax+b),f(x,y)) \Rightarrow  0 \neq_{L\not \subset C} f(x,ax+b)\in k[x]$, tiene grado $\le m\Rightarrow $ existen a lo más $m$ posibles valores de $x$ tal que $f(x)=0$.

\textbf{Ejercicio 1.13: } (a) $\{(x,y) \in \mathbb{A}^2(\mathbb{R}) : y= sin(x)\}$, hay que probar que no es un conjunto algebraico.

\underline{\textit{Solución: }}

Si fuera algebraico, entonces sería $C=V(S), S\subseteq k[x_1\dots, x_n]$. $C\cap V(y)=\{(\pi k,0): k\in \mathbb{Z} \} \subseteq L=V(y)$ recta afin. Dicho conjunto es infinito y distinto del total, entonces no es conjunto algebraico, y la intersección de algebraicos debería ser algebraica.

\textbf{Ejercicio 1.25: } Probar que $V(y-x^2)\subseteq \mathbb{A}^2(\mathbb{C})$ es irreducible e $IV(y-x^2)=<y-x^2>$.

\underline{\textit{Solución: }}

\begin{nota}
Sabemos que dado un conjunto algebraico, era irreducible si y solo si $I(V)$ era ideal primo.
\end{nota}

Basta ver que $IV(y-x^2)=<y-x^2>$ y que es primo.  $V(y-x^2)=\{(t,t^2): t\in \mathbb{C} \}$. $\supseteq $ la tenemos ya, probamos la contraria. Sea $f\in IV(y-x^2) \Rightarrow f(t,t^2)=0, \forall t \in \mathbb{C}$. Por otro lado, $f=q(x,y)(y-x^2)+r(x) \Rightarrow r(t)=0, \forall t\in \mathbb{C} \Rightarrow r=0 \Rightarrow f=q(y-x^2)\in <y-x^2>$, y se tiene la igualdad.

Falta ver que es ideal primo, $\frac{k[x,y]}{<y-x^2>} \cong k[x]$, $k[x]\rightarrow \frac{k[x,y]}{<y-x^2>}$ es isomorfismo. La sobreyectividad se tiene porque $\bar{f(x,y)}=\bar{f(x,x^2)}, f(x,x^2)=g(x)\in k[x]$.

\section{Módulos; Condiciones de finitud}

Sea $\mathbb{A}$ anillo conmutativo y con elemento  unidad.


\index{\texttt{A-módulo}}
\begin{Def}
Un \textbf{A-módulo} $M$ es un grupo abeliano $(M,+)$ dotada de una operación $\cdot: A\times M \rightarrow M$, que verifica:

\begin{enumerate}
\item $(a+b)m=am+bm, \forall a , b \in A, \forall m\in M$.
\item $a(m+m')=am+am', \forall a \in A , \forall m, m'\in M$.
\item $a(bm)=(ab)m$.
\item $1_A m = m, \forall m \in M$.
\end{enumerate}
\end{Def}

\textbf{Ejemplo: }

\begin{itemize*}
\item Si $A=\mathbb{Z}$, entonces un $\mathbb{Z}-$módulo es lo mismo que un grupo abeliano $a\in \mathbb{Z},m\in M, am=m+\cdots +m$.
\item $A=k$ cuerpo, entonces un $k-$módulo es lo mismo que un k-espacio vectorial.
\item Un anillo $A$ también es un $A-$módulo.
\end{itemize*}


\index{\texttt{Submódulo}}
\begin{Def}
Si $M$ es un $A-$módulo, un subgrupo abeliano $N\subset M$ es un \textbf{submódulo} si $am\in N, \forall a \in A, \forall m \in N$.
\end{Def}

\textbf{Ejemplo: }

Los submódulos de un anillo $A$ son sus ideales.


\index{\texttt{A-Módulo f.g}}
\begin{Def}
Sea $M$ un $A-$ módulo, diremos que un submódulo $N\subset M$ es \textbf{finitamente generado} ($f.g.$) si $\exists m_1,\dots, m_r\in N$ tales que $N=Am_1+\cdots +Am_r$ (es decir, $\forall n \in N, \exists a_1,\dots, a_r\in A$ tales que $n=\sum_{i=1}^ra_im_i$).
\end{Def}

Sea $R\subset S$ una extensión de anillos, se pueden considerar ciertas condiciones de finitud de $R$ sobre $S$. ($S$ es un $R-$módulo).

\index{\texttt{R-Álgebra}}
\begin{itemize*}
\item $S$ se dice \textbf{$R-$ álgebra} $f.g.$ si $\exists v_1,\dots,v_n\in S$ tales que $S=R[v_1,\dots,v_n]$. Es decir, $\exists \varphi: R[x_1,\dots,x_n]\rightarrow S$ sobreyectivo (morfismo), $f(x_1,\dots,x_n)\rightarrow f(v_1,\dots,v_n)$. Por el primer teorema de isomorfía $\frac{R[x_1,\dots,x_n]}{ker\varphi}\cong Im\varphi =S$. (No es lo mismo que un anillo de polinomios).

\item ¿Es $S$ un $R-$ módulo finitamente generado?

\item Si $R=k\subset S=L$ es extensión de cuerpos y $L$ es una $k-$álgebra $f.g.$, se dice que $L$ es una extensión finitamente generada de $k$. 

$$v_i\in L, L=k[v_1,\dots,v_r]=k(v_1,\dots,v_r)$$

\item Una extensión $k\subset L$ de cuerpos se dice que es $f.g.$ si $\exists v_1,\dots,v_m\in L$ tales que $L=k(v_1,\dots,v_n)$.
\end{itemize*}

\textbf{Ejemplos: }

\begin{itemize*}
\item $A$ anillo, $S=A[x]$; $A\subset S$, y podemos ver que tomando $\varphi$ la propia inclusión se tiene que $S$ es una $A-$álgebra $f.g.$.

No es un $A-$módulo finitamente generado. Reducción al absurdo: $\exists f_1,\dots, f_r\in A[x]$ tal que $S=Af_1+\dots+Af_r, \forall f\in S\Rightarrow deg(f)\le max\{deg(f_i)\} \rightarrow \leftarrow$.

\item $k\subseteq k(x)$ es una extensión de cuerpos f.g. pero no es una $k-$álgebra finitamente generada, aunque sí es un k-álgebra por el hecho de contener el anillo $k$.

\item Un ejemplo de k-álgebra finitamente generada es $\frac{R[x_1,\dots,x_n]}{I}=R[\bar{x}_1,\dots,\bar{x}_n]$
\end{itemize*}

\textbf{Cuestión a entregar viernes 14 Octubre:} Ejercicios 1.44, 1.45.


Sea $R\subset S$ una extensión de anillos tal que $S$ sea $R-$ módulo finitamente generado. ¿Es $S$ una $R-$álgebra finitamente generada?

Sí, $S=\sum Rv_i \subset R[v_1,\dots,v_n]\subseteq S \Rightarrow S=R[v_1,\dots, v_n]$.

$R[v_1,\dots,v_n]= \{\sum r_\alpha v_1^{\alpha_1}\cdots v_n^{\alpha_n}: r_\alpha \in R\}$.

\section{Elementos enteros y dependencia entera}

Sea $R\subset S$ extensión de anillos.

\index{\texttt{Elemento entero}}
\begin{Def}
Un elemento $s\in S$ es \textbf{entero} sobre $R$ si $\exists 0\neq f\in R[x]$ mónico tal que $f(s)=0$.
\end{Def}

\begin{nota}
Si $R$ y $S$ son cuerpos, un elemento $s\in S$ entero sobre $R$ es lo mismo que un elemento algebraico sobre $R$.
\end{nota}

\textbf{Ejemplo: }

Sea $r\in \mathbb{Q}$ tal que $r$ sea entero sobre $\mathbb{Z}$.

$r=\frac{a}{b}, gcd(a,b)=1$.

$r^n+a_{n-1}r^{n-1}+\dots,a_0=0$ en $\mathbb{Q}, a_i\in \mathbb{Z} \Rightarrow a^n=-a_{n-1}ba^{n-1}+\dots,a_0b^n=0=b[a_{n-1}a^{n-1}+\dots+a_0b^{n-1}]\Rightarrow b|a^n \Rightarrow b=1 \vee b=-1$.

Luego, $r=\frac{a}{b} \in \mathbb{Z}$.

\begin{Prop}
$R\subset S$ anillos, $v\in S$. Son equivalentes: 
\begin{enumerate}
\item $v$ entero sobre $R$.
\item $R[v]$ es un $R-$módulo finitamente generado.
\item $\exists R'\subseteq S$ subanillo de $S$ tal que $R[v]\subseteq R'$ y $R'$ es un $R-$módulo finitamente generado.
\end{enumerate}
\end{Prop}

\begin{Dem}
\framebox{$1\Rightarrow 2$} $\exists n\ge 1, v^n+a_{n-1}v^{n-1}+\dots+a_0=0, a_i\in R \Rightarrow v^m=-\sum_{i=0}^{n-1}a_iv^i$.

$R[v]=\{ \sum_{m=0}^k a_mv^m: a_i\in R,i=0,\dots, k\}=Rv^0+Rv+\dots+Rv^{n-1}$ $R-$módulo.

\framebox{$2\Rightarrow 3$} $R'=R[v]$.

\framebox{$3\Rightarrow 1$} $R[v]\subset R' \subset S$, $R'$ es un $R-$módulo finitamente generado. $R'=Rv_1+\dots+Rv_n$, $v_i\in R'$.

$v\in R'\Rightarrow v\cdot v_i\in R' \Rightarrow \sum_{j=1}^n \delta_{ij}v_j= v\cdot v_i=\sum_{j=1}^n a_{ij}v_j \Rightarrow \sum_{j=1}^n(\delta_{ij}v\cdot v-a_{ij})v_j=0 \Rightarrow (Iv-A) (v_1\cdots v_n)'=0 \Rightarrow \underbrace{det(IV-A)}_{\in R[v]\subset R}(v_1\cdots v_n)'=\vec{0} \Rightarrow_{I_{R'}=\sum b_iv_i} det(IV-A)=0 \Rightarrow v$ es entero sobre $R$

\end{Dem}

\begin{nota}
$\delta$ es la delta que vale 1 o 0, kronickle.
\end{nota}

\textbf{Ejercicios viernes 14 Oct:} 1.38,1.44,1.45,1.49,1.50.

$R\subset S, v\in S$, $v$ entero sobre $R \Leftrightarrow R[v]$ es $R-$módulo finitamente generado.

\textbf{Ejemplo: }

$R=\mathbb{Z}\subset S=R, v=\sqrt{2}$ es entero sobre $\mathbb{Z}$, ya que $v^2-2=0$, de lo que se deduce que $\mathbb{Z}[\sqrt{2}]$ es $\mathbb{Z}-$módulo, grupo abeliano, finitamente generado. En concreto $\mathbb{Z}[\sqrt{2}]=\mathbb{Z}\cdot 1+\mathbb{Z}\sqrt{2}$.


\begin{Cor}
Sea $R\subset S$ extensión de anillos $\Rightarrow $ el conjunto $R'$ de los elementos de $S$ que son enteros sobre $R$ es un subanillo de $S$ que contiene a $R$.
\end{Cor}

\begin{Dem}
Todo elemento $r\in R$ es entero sobre $R$ ya que $f=x-r\in R[x]$ es mónico no nulo y $f(r)=0 \Rightarrow R\subseteq R'\subseteq S$. 

Hay que probar que $\forall a,b\in R'$ se cumple $a-b\in R'$, $ab\in R'$.

Como $a$ es entero sobre $R\Rightarrow R[a]$ es un $R-$módulo finitamente generado, y análogo para $b$. Si es $b$ es entero sobre $R$, entonces $b$ es entero para $R[a]\Rightarrow R[a,b]$ es $R[a]-$módulo finitamente generado $\Rightarrow R[a,b]$ es $R-$módulo finitamente generado. 

$a-b,ab \in R[a,b] \Rightarrow R[a-b],R[ab]\subset R[a,b]\subset S \Rightarrow a-b,ab$ enteros sobre $R$, es decir, $a-b,ab\in R'$.
\end{Dem}

\section{Extensiones de cuerpos}

Sea $k\subset L$ extensión de cuerpos finitamente generado, $L=k(v_1,\dots,v_n)$, para ciertos $v_i\in L$.

Si $n=1$, $L=k(v)=\{\frac{f(v)}{g(v)}:f,g\in k[x],g(v)\neq 0 \}$.

Sea $\varphi:k[x]\rightarrow L, f(x)\rightarrow f(v)$ morfismo de anillos. Por el primer teorema de isomorfia se tiene $\frac{k[x]}{ker \varphi}\cong k[v]\subseteq L$. 

Caso 1: $ker \varphi = 0 \Rightarrow k[x] \cong k[v]\neq k(v)=L$, porque $k[x]\neq k(x)$. Entonces $L$ no es una $k-$álgebra finitamente generada, y tampoco es $k-$espacio vectorial de dimensión finita, y $v$ no es algebraico sobre $k$.

Caso 2: $ker\varphi \neq 0 \Rightarrow ker \varphi =<f>$,  omo $k[v]$ es dominio de integridad, es un ideal primo $\Rightarrow f\in k[x]$ irreducible $\Rightarrow ker \varphi =<f>$ maximal (Por un ejercicio anterior). Entonces, $\frac{k[x]}{ker\varphi}$ es un cuerpo, luego $k[v]$ también es un cuerpo $\Rightarrow k(v)=k[v]=L$ es $k-$álgebra finitamente generada.

Por otro lado, como $f\in ker\varphi \Rightarrow f(v)=0$, como $k$ es un cuerpo, puedo suponer $f\neq 0$ mónico, y se tiene que $v$ es entero (algebraico) sobre $k$.

\begin{nota}
$k(v)$ cuerpo de fracciones, menor cuerpo que lo contiene.
\end{nota}

Entonces por la proposición anterior, se tiene que $L$ es $k-$módulo finitamente generado, es decir, $k-$espacio vectorial de dimensión finita.

De manera más general tenemos lo siguiente: 

\begin{Prop}
\textbf{(Zariski)} Sea $k\subset L$ extensión de cuerpos y $L$ es una $k-$álgebra finitamente generada $\Rightarrow$ $L$ es un $k-$espacio vectorial de dimensión finita y, por tanto, $L$ es algebraico sobre $k$. ($k\subset L$ extensión algebraica).
\end{Prop}

\begin{Dem}
Se demuestra por inducción, $L=k[v_1,\dots,v_n]$ $k-$álgebra finitamente generada por $v_1,\dots,v_n\in L$.

Por el razonamiento anterior lo tenemos probado para $n=1$, lo suponemos cierto para $n-1$, y lo probamos para $n$.

$L=k[v_1,\dots,v_n]=k'[v_2,\dots,v_n]$ con $k'=k(v_1)$. Como $L$ es $k'-$ álgebra finitamente generada, se tiene que $L$ es un $k'-$espacio vectorial de dimensión finita. 

Por otro lado, si consideramos $k'=k(v_1]), k\subset k'$ es una extensión. Si $v_1$ es algebraico sobre $k$, entonces $k'$ es $k-$ espacio vectorial de dimensión finita. 

Por ambas cosas, se tendrá que $L$ es $k-$espacio vectorial de dimensión finita.

Supongamos que $v_1$ no es algebraico sobre $k$. Entonces, $k'=k(v_1) \cong k(x)$. 

Como $(k'(v_i)\subset)L$ es $k'-$espacio vectorial de dimensión finita. Entonces, $v_2,\dots,v_n$ son enteros, y por tanto, algebraicos sobre $k'=k(v_1) \Rightarrow (v_i)^{n_i}+a_{in}(v_1)^{n_i-1}+\dots=0,a_{ij}\in k'=k(v) \Rightarrow \exists b\in k[v_1]$ tal que $ba_{ij}\in k[v_1] \Rightarrow (bv_i)^{n_i}+a_{in_i}b(bv_i)^{n_i-1}+\dots =0 \Rightarrow bv_i$ enteros sobre $k[v_1]$.

$\Rightarrow k[v_1,bv_2,\dots,bv_n]=k'[bv_2,\dots,bv_n]$ es $k[v_1]-$módulo finitamente generado. $\forall f\in k[v_1,\dots,v_n]=L, \exists N\ge 0$ tal que $b^Nf \in k[v_1,bv_2,\dots,bv_n]$ esentero sobre $k[v_1]$. Como en particular, $k(v_1)\subset L$, entonces $\forall g\in k(v_1) \cong k(x),\exists N\ge 0$ tal que $b^Ng$ es entero sobre $k[v_1]\cong k[x]$, por dos ejercicios propuestos (1.44,1.45), ésto no es posible. \qed
\end{Dem}

\textbf{Demostración del teorema débil de los ceros de Hilbert:}

$I\subset m\subseteq k[\vec{x}]$, $m$ ideal maximal $\Rightarrow v(m)\subseteq V(I)$. Hay que probar que $v(m)\neq \emptyset $. 

Se considera la extensión $k\subseteq L=\frac{k[x_1,\dots,x_n]}{m}$. 

Si $k=L \Rightarrow \forall i=1,\dots, n, \bar{x}_i=\bar{a}_i, a_i\in k \Rightarrow x_i-a_i\in m \Rightarrow <x_1-a_1,\dots,x_n-a_n>\subseteq m$, es un ideal maximal pues $\frac{k[\vec{x}]}{<x_1-a_1,\dots,x_n-a_n>} \cong k \Rightarrow m=<x_1-a_1,\dots,x_n-a_n> \Rightarrow v(m)=\{ (a_1,\dots,a_n)\} \neq \emptyset $ .

Hay que probar que $k=L$, $k\subseteq L=\frac{k[x_1,\dots,x_n]}{m}$ cuerpo y k-álgebra finitamente generada $\xrightarrow{\text{Prop}} k\subseteq L$ extensión algebraica. $k$ algebraicamente cerrado $\Rightarrow L=k$.
\vspace{2mm}


\textbf{Problema 1.25:} (b) $V(y^4-x^2,y^4-x^2y^2+xy^2-x^3)$

\underline{\textit{Solución: }}

$(y^2-x)(y^2+x)$.

Si $x=y^2\rightarrow y^4-y^6+y^4-y^6 =0 \rightarrow (y=0 \rightarrow x=0 \rightarrow (0,0)) \text{ ó } (y=+-1 \rightarrow x=1 (1,+-1))$.

$x=-y^2$ no añade ninguna condición.

$W=V(x+y^2)\cup \{(1,1)\}\{(1,-1)\}$ tres componentes irreducibles.

\vspace{2mm}

\textbf{Problema 1.26: } \textit{Probar que $F=y^2+x^2(x-1)^2$ es irreducible en $\mathbb{R}[x,y]$, pero $V(F)$ es reducible.}

\underline{\textit{Solución: }}

$F=(y-ix(x-1))(y+ix(x-1))$ factorización en $\mathbb{C}[x,y]$, es un dominio de factorización única, y vemos que eso implica que es irreducible en $\mathbb{R}[x,y]$ pues multiplicando por escalares, mantenemos algún elemento complejo.

$y^2+x^2(x-1)^2=0 \Rightarrow y=0 \wedge [x=0 \vee x=1] \Rightarrow V(F)=\{(0,0),(1,0)\}.$

\vspace{2mm}
\textbf{Problema 1.27: } \textit{Sean $V,W\subseteq \mathbb{A}^n$ conjuntos algebraicos, $V\subseteq W$. Probar que cada componente irreducible de $V$ está contenida en una componente irreducible de $W$.}

\underline{\textit{Solución: }}

$V_i\subseteq W \Rightarrow V_i=(V_i\cap W_i)\cup \cdots \cup (V_i\cap W_r) \Rightarrow V_i= V_i\cap W_i \vee V_i = \cup_{j=2}^r(V_i\cap W_j) \Rightarrow \exists j  | v_i=W_j\cap V_i \Rightarrow V_i\subseteq W_j$.

\vspace{2mm}
\textbf{Problema 1.28: } Si $V=V_1\cup \cdots \cup V_r$ es la descomposición de $V$ en conjuntos algebraicos irreducibles hay que probar que $V_i \not \subset \cup_{j\neq i}V_j$.

\underline{\textit{Solución: }}

Reducción al absurdo: Si $V_i\subseteq \cup_{j\neq i}V_j \Rightarrow \exists j\neq i | V_i \subseteq V_j \rightarrow \leftarrow $.

\vspace{2mm}
\textbf{Problema 1.29: } $k$ infinito $\Rightarrow \mathbb{A}^n(k)$ irreducible.

\underline{\textit{Solución: }}

$I(\mathbb{A}^n)= \{F\in k[\vec{x}] | F(a_1,\dots,a_n)=0\}$ pero por un problema $F=0 \Rightarrow =<0>$ que es primo, y se tiene que $\mathbb{A}^n $ irreducible.

\vspace{2mm}
\textbf{Problema 1.32:}

\underline{\textit{Solución: }}
$I=<x^2+1>\subset R[x], V(I)=\emptyset $.

$J=<x^2(x-1)^2+y^2>\subseteq \mathbb{R}[x,y]$ primo, $V(J)=\{(0,0)\}\cup \{(1,0)\}$ reducible.

\vspace{2mm}
\textbf{Problema 1.33:} \textit{(a) Descomponer $V(x^2+y^2-1,x^2-z^2-1)\subseteq \mathbb{A}^3(\mathbb{C})$.}

\underline{\textit{Solución: }}

(a) $f-g=(y-iz)(y+iz) \Rightarrow W=V(y+iz,x^2-z^2-1)\cup V(y-iz,x^2-z^2-1)$. Para ver que es irreducible, como estamos en un cuerpo algebraicamente cerrado, basta ver que $IV(\cdot)=\sqrt{<y+iz,x^2-z^2-1>}$. 

$\frac{\mathbb{C}[x,y,z]}{<y+iz,x^2-z^2-1>} \sim \frac{\mathbb{C}[x,z]}{<x^2-z^2-1>}$ ddi. Lo prueba el ejercicio siguiente.

\textbf{Problema 1.34: } \textit{Sea $R$ un dominio de factorización única:}
\begin{enumerate}
\item \textit{Probar que un polinomio mónico de grado 2 o 3 en $R[X]$ es irreducible si y sólo si no tiene raices en $R$.}
\item El polinomio $x^2-a$ en $R[x]$ es irreducible $\Leftrightarrow a=b^2$ para algún $b\in R$. 
\end{enumerate}

\underline{\textit{Solución: }}

\begin{enumerate}
\item $f\in R[x], deg(f)=2 \vee deg(f)=3, f$ mónico. $f$ reducible $\Rightarrow f=(cx+d)g$ con $g$ de grado 1 o 2. $\Rightarrow $ el coeficiente lider de $g$ $b$, $cb=1 \Rightarrow  c$ unidad en $R$ $\Rightarrow x=c^{-1}d\in R$ y es raíz de $f$.

$f$ tiene raíz $a\in R \Rightarrow f(a)=0, f=q(x-a)+r \Rightarrow r=0 \Rightarrow f=q(x-a)$ y por lo tanto, es reducible.

\item $\exists b \in R$ tal que $b^2-a=0$
\end{enumerate}

\textbf{Problema 1.37: } Sea $k$ un cuerpo, $F\in k[x]$ polinomio de grado $n>0$. Hay que probar que $\{\bar{1},\dots,\bar{X}^{n-1}\}$ base de $\frac{k[\vec{x}]}{<F>}$.

\underline{\textit{Solución: }}

Puedo suponer $F$ mónico ya que $<F>=<cF>, \forall c \in k$.

$\bar{F}=\bar{0}$ en $A$ $\Rightarrow \bar{X}^n=\bar{X^n}=\bar{-a_{n-1}x^{n-1}-\dots -a_0}$ k espacio vectorial generado por $B$.

Por inducción, todas las potencias de $\bar{X}$ se pueden expresar de esa manera. 

Veamos que son linealmente independientes: Supongo $\exists \lambda_i\in k$ tal que $\sum_{i=0}^{n-1}\lambda_i\bar{x}^i=\bar{0}$ en $A \Rightarrow \sum_{i=0}^{n-1}\lambda_ix^i\in <F> \Rightarrow \sum_{i=0}^{n-1}\lambda_ix^i=FQ \rightarrow \leftarrow \Rightarrow \lambda_i=0, \forall i$.

